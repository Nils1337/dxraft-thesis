\chapter{Fazit}

In dieser Arbeit wurde ein verteilter Koordinierungsdienst namens DXRaft entwickelt und implementiert. Dafür wurden zunächst die Anforderungen an das System formuliert. Dann wurde analysiert, wie ein solches System implementiert werden kann. Dabei wurde festgestellt, dass für die formulierten Anforderungen ein Einigungsalgorithmus implementiert werden muss. Deshalb wurden verschiedene Einigungsalgorithmen vorgestellt und analysiert. Dabei wurde Raft als zu implementierender Algorithmus ausgewählt. \\
Es wurden die Probleme und die Lösungen bei der Implementierung erläutert. Dabei wurde das Konzept der Implementierung erläutert. Es wurde erklärt, wie das System Nachrichten austauscht und wie das Bootstrapping abläuft. Außerdem wurde ein Session-Mechanismus und eine Möglichkeit der Änderung der Clusterkonfiguration während des laufenden Betriebs implementiert. Des Weiteren wurde erklärt, wie das System das Log effizient auf der Festplatte persistiert und wie die Schnittstelle des Clients aufgebaut ist. Schließlich wurde erläutert, wie durch Tests sichergestellt wurde, dass das System korrekt funktioniert. \\
 Dann wurde das System evaluiert, in dem es mit anderen ähnlichen in der Industrie verwendeten Systemen verglichen wurde. Dabei wurde festgestellt, dass das implementierte System bei der durchschnittlichen Latenz und beim Durchsatz etwas bessere Ergebniss liefert als die anderen Systeme. Auch den Ausfall eines Leaders kann das System im Vergleich zu den anderen System gut kompensieren und wählt sehr schnell einen neuen Leader um mit dem Bearbeiten von Anfragen fortzufahren. Jedoch ist die Latenz nicht so stabil wie bei den anderen Systemen, da die maximale Latenz höher ist. Deshalb wurde noch untersucht, wie die Latenz in DXRaft zustande kommt. Dabei wurde festgestellt, dass vor allem die Kommunikation die Höhe der Latenz bedingt. Es wurde sowohl die Kommunikation zwischen Client und Server als auch die Kommunikation zwischen der Servern als Engpässe identifiziert. \\
Schließlich wurden Verbesserungsmöglichkeiten für weitere Arbeiten aufgezeigt. Diese umfassen eine Log-Kompaktierung, das Recovery von ausgefallenen Servern, das Testen auf \textit{Linearizability} und Performanceverbesserungen durch Optimierung der Thread-Architektur und des Nachrichtenaustausches.\\